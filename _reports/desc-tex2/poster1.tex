%% BEGIN poster1.tex
%%
%% Sample for poster.tex/poster.sty.
%% Run with LaTeX, with or without the NFSS.
%% You might have problems with missing fonts.
%%
%% See below if using A4 paper.

\documentstyle{article}

\input poster  % Input here in case poster.sty not installed.

\mag\magstep5  % Magnification of 1.2^5 (roughly 2.5)
               % Use "true" dimensions below for magnified values.

\begin{document}

%% Add  paperwidth=210mm,paperheight=297mm  if using A4 paper:

\begin{Poster}[vcenter=true,hcenter=true]
\setlength{\fboxsep}{.8truein}%
\setlength{\fboxrule}{.1truein}%
\fbox{\begin{minipage}{11.1truein}

\begin{center}
  \bf ON SOME \boldmath$\Pi$-HEDRAL SURFACES IN QUASI-QUASI SPACE
\end{center}
\begin{center}
  CLAUDE HOPPER, Omnius University
\end{center}

There is at present a school of mathematicians which holds that the
explosive growth of jargon within mathematics is a deplorable trend.  It
is our purpose in this note to continue the work of
Redheffer~\cite{redheffer} in showing how terminology itself can lead to
results of great elegance.

I first consolidate some results of Baker~\cite{baker} and
McLelland~\cite{mclelland}.  We define a class of connected snarfs as
follows: $S_\alpha=\Omega(\gamma_\beta)$.  Then if
$B=(\otimes,\rightarrow,\theta)$ is a Boolean left subideal, we have:
$$
\nabla S_\alpha=\int\int\int_{E(\Omega)}
B(\gamma_{\beta_0},\gamma_{\beta_0})\,d\sigma d\phi d\rho
-\frac{19}{51}\Omega.
$$
Rearranging, transposing, and collecting terms, we have:
$\Omega=\Omega_0$.

The significance of this is obvious, for if $\{S_\alpha\}$ be a class of
connected snarfs, our result shows that its union is an utterly
disjoint subset of a $\pi$-hedral surface in quasi-quasi space.

We next use a result of Spyrpt~\cite{spyrpt} to derive a property of
wild cells in door topologies.  Let $\xi$ be the null operator on a door
topology, $\Box$, which is a super-linear space.  Let $\{P_\gamma\}$ be
the collection of all nonvoid, closed, convex, bounded, compact,
circled, symmetric, connected, central, $Z$-directed, meager sets in
$\Box$.  Then $P=\cup P_\gamma$ is perfect.  Moreover, if $P\neq\phi$,
then $P$ is superb.

\smallskip
{\it Proof.}  The proof uses a lemma due to
Sriniswamiramanathan~\cite{srinis}.  This states that any unbounded
fantastic set it closed.  Hence we have
$$
\Rightarrow P\sim\xi(P_\gamma)-\textstyle\frac{1}{3}.
$$

After some manipulation we obtain
$$
\textstyle\frac{1}{3}=\frac{1}{3}
$$
I have reason to believe~\cite{russell} that this implies $P$ is perfect.
If $P\neq\phi$, $P$ is superb.  Moreover, if $\Box$ is a $T_2$ space, $P$
is simply superb.  This completes the proof.

Our final result is a generalization of a theorem of Tz, and
encompasses some comments on the work of Beaman~\cite{beaman} on the
Jolly function.

Let $\Omega$ be any $\pi$-hedral surface in a semi-quasi space.  Define
a nonnegative, nonnegatively homogeneous subadditive linear functional
$f$ on $X\supset\Omega$ such that $f$ violently suppresses $\Omega$.
Then $f$ is the Jolly function.

\smallskip
{\it Proof.}  Suppose $f$ is not the Jolly function.  Then
$\{\Lambda,\mbox{@},\xi\}\cap\{\Delta,\Omega,\Rightarrow\}$ is void.  Hence
$f$ is morbid.  This is a contradiction, of course.  Therefore, $f$ is
the Jolly function.  Moreover, if $\Omega$ is a circled husk, and
$\Delta$ is a pointed spear, then $f$ is uproarious.

\small
\begin{center}
\bf References
\end{center}
\def\thebibliography#1{%
  \list
 {\bf\arabic{enumi}.}{\settowidth\labelwidth{\bf #1.}\leftmargin\labelwidth
 \advance\leftmargin\labelsep
 \usecounter{enumi}}
 \def\newblock{\hskip .11em plus .33em minus .07em}
 \sloppy\clubpenalty4000\widowpenalty4000
 \sfcode`\.=1000\relax}
\begin{thebibliography}{9}
\bibitem{redheffer}
R. M. Redheffer, A real-life application of mathematical symbolism,
this {\it Magazine}, 38 (1965) 103--4.
\bibitem{baker}
J. A. Baker, Locally pulsating manifolds, East Overshoe Math. J., 19
(1962) 5280--1.
\bibitem{mclelland}
J. McLelland, De-ringed pistons in cylindric algebras,
Vereinigtermathematischerzeitung f\"ur Zilch, 10 (1962) 333--7.
\bibitem{spyrpt}
Mrowclaw Spyrpt, A matrix is a matrix is a matrix, Mat. Zburp., 91
(1959) 28--35.
\bibitem{srinis}
Rajagopalachari Sriniswamiramanathan, Some expansions on the Flausgloten
Theorem on locally congested lutches, J. Math. Soc., North Bombay, 13
(1964) 72--6.
\bibitem{russell}
A. N. Whitehead and B. Russell, Principia Mathematica, Cambridge
University Press, 1925.
\bibitem{beaman}
J. Beaman, Morbidity of the Jolly function, Mathematica Absurdica, 117
(1965) 338--9.
\end{thebibliography}
\end{minipage}}%
\end{Poster}

\end{document}
%% END poster1.tex

